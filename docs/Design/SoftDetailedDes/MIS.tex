\documentclass[12pt, titlepage]{article}

\usepackage{amsmath, mathtools}

\usepackage[round]{natbib}
\usepackage{amsfonts}
\usepackage{amssymb}
\usepackage{graphicx}
\usepackage{colortbl}
\usepackage{xr}
\usepackage{hyperref}
\usepackage{longtable}
\usepackage{xfrac}
\usepackage{tabularx}
\usepackage{float}
\usepackage{siunitx}
\usepackage{booktabs}
\usepackage{multirow}
\usepackage[section]{placeins}
\usepackage{caption}
\usepackage{fullpage}

\hypersetup{
bookmarks=true,     % show bookmarks bar?
colorlinks=true,       % false: boxed links; true: colored links
linkcolor=red,          % color of internal links (change box color with linkbordercolor)
citecolor=blue,      % color of links to bibliography
filecolor=magenta,  % color of file links
urlcolor=cyan          % color of external links
}

\usepackage{array}

%% Comments
\newif\ifcomments\commentstrue

\ifcomments
\newcommand{\authornote}[3]{\textcolor{#1}{[#3 ---#2]}}
\newcommand{\todo}[1]{\textcolor{red}{[TODO: #1]}}
\else
\newcommand{\authornote}[3]{}
\newcommand{\todo}[1]{}
\fi

\newcommand{\wss}[1]{\authornote{blue}{SS}{#1}}
\newcommand{\bmac}[1]{\authornote{red}{BM}{#1}}

\newcommand{\progname}{SWHS}

\begin{document}

\title{Module Interface Specification for Data Center Minimization Cost Analysis}

\author{Ning Wang}

\date{\today}

\maketitle

\pagenumbering{roman}

\newpage

\tableofcontents

\newpage

\section{Symbols, Abbreviations and Acronyms}

    See SRS Documentation at \url{https://github.com/KarenKarenWang/cas741_project1/tree/main/docs/SRS}

\pagenumbering{arabic}

\section{Introduction}

The following document details the Module Interface Specifications for the
implemented modules in a program simulating a Minimization with
Phase Change Material.  It is intended to ease navigation through the program
for design and maintenance purposes.

Complementary documents include the System Requirement Specifications
and Module Guide.  The full documentation and implementation can be
found at \url{https://github.com/KarenKarenWang/cas741_project1/tree/main/docs/SRS}.

The specification is given in terms of functions, rather than sequences.  For
instance, the power loss along transmission is given as a function of distances
($\mathbb{R} \rightarrow \mathbb{R})$, not as a sequence ($\mathbb{R}^n$).  This
approach is more straightforward for the specification, but in the
implementation stage, it will likely be necessary to introduce a
sequence, assuming that a numerical solver is used for the system of linear programming.

\section{Notation}

The structure of the MIS for modules comes from \citet{HoffmanAndStrooper1995},
with the addition that template modules have been adapted from
\cite{GhezziEtAl2003}.  The mathematical notation comes from Chapter 3 of
\citet{HoffmanAndStrooper1995}.  For instance, the symbol := is used for a
multiple assignment statement and conditional rules follow the form $(c_1
\Rightarrow r_1 | c_2 \Rightarrow r_2 | ... | c_n \Rightarrow r_n )$.

The following table summarizes the primitive data types used by \progname. 

\begin{center}
\renewcommand{\arraystretch}{1.2}
\noindent 
\begin{tabular}{l l p{7.5cm}} 
\toprule 
\textbf{Data Type} & \textbf{Notation} & \textbf{Description}\\ 
\midrule
character & char & a single symbol or digit\\
integer & $\mathbb{Z}$ & a number without a fractional component in (-$\infty$, $\infty$) \\
natural number & $\mathbb{N}$ & a number without a fractional component in [1, $\infty$) \\
real & $\mathbb{R}$ & any number in (-$\infty$, $\infty$)\\
\bottomrule
\end{tabular} 
\end{center}

\noindent
The specification of \progname \ uses some derived data types: sequences, strings, and
tuples. Sequences are lists filled with elements of the same data type. Strings
are sequences of characters. Tuples contain a list of values, potentially of
different types. In addition, \progname \ uses functions, which
are defined by the data types of their inputs and outputs. Local functions are
described by giving their type signature followed by their specification.

\section{Module Decomposition}

The following table is taken directly from the Module Guide document for this project.

\begin{table}[h!]
\centering
\begin{tabular}{p{0.3\textwidth} p{0.6\textwidth}}
\toprule
\textbf{Level 1} & \textbf{Level 2}\\
\midrule

{Hardware-Hiding Module} & ~ \\
\midrule

\multirow{7}{0.3\textwidth}{Behaviour-Hiding Module} 
& Input Parameters Module\\
&Input Verification Module\\
& Output Format Module\\
& Output Verification Module\\
& Control Module\\
& Specification Parameters Module\\
\midrule

\multirow{3}{0.3\textwidth}{Software Decision Module} 
& Optimization Module\\
&Sequence Data Structure Module\\
&Plotting Module\\
\bottomrule

\end{tabular}
\caption{Module Hierarchy}
\label{TblMH}
\end{table}




\section{MIS of Control Module} \label{Main}

\subsection{Module}

main

\subsection{Uses}

Parameter (Section~\ref{Parameters}), Temperature
(Section~\ref{Temperature}), Optimization
(Section~\ref{ODE}), Energy (Section~\ref{Energy}), verify\_output (Section~\ref{VerifyOutput}), plot
(Section~\ref{Plot}), output (Section~\ref{Output})

\subsection{Syntax}

\subsubsection{Exported Access Programs}

\begin{center}
\begin{tabular}{p{2cm} p{4cm} p{4cm} p{2cm}}
\hline
\textbf{Name} & \textbf{In} & \textbf{Out} & \textbf{Exceptions} \\
\hline
main & - & - & - \\
\hline
\end{tabular}
\end{center}

\subsection{Semantics}

\subsubsection{State Variables}

None

\subsubsection{Access Routine Semantics}

\noindent main():
\begin{itemize}
\item transition: Modify the state of Param module and the environment variables
  for the Plot and Output modules by following these steps\\
\end{itemize}

\noindent Get (filenameIn: string) and (filenameOut: string) from user\\

\noindent load\_params(filenameIn)\\

\noindent \#\textit{Find minimization cost function} ($C_T$,
  $C_R$, $C_G$, $C_g$, $C_r$, $L_T$, $R_T$, $P_T$), \textit{and transmission loss} ($T_L$) \textit{and distances of data centers with power station} ($d_i$)\\

\noindent $A_{eq} \cdot x = b_{eq}$ := \text{solve}(\text{linp\_optimization}, $L_T$,
$P_T$, $R_T$, $C_T$, $C_R$, $C_G$,)\\

\noindent $L_T$ := solve(linp\_optimization,
$R_i^N$, $P_i^N$)\\


\noindent \#\textit{find transmission loss along distances}\\
\\
\indent $T_L = (d_i \cdot 0.03 \cdot P_i)^{N}$\\

\noindent \#\textit{Power distribution}\\
\\
\indent $L_i$ = $R_i$ + $P_i$ \\

\noindent \#\textit{Output calculated values to a file and to a plot.  Verify
 the sum of calculated values as less than total power consumption.}\\

\noindent verify\_output($L_T$, $R_T$, $P_T$)\\

\noindent plot($L_i$, $d_i$)\\

\noindent output(filenameOut, $L_i$, $d_i$, $R_i$, $P_i$, $C_T$)\\

\newpage

\section{MIS of Input Parameters Module} \label{Parameters}

The secrets of this module are the data structure for input parameters, how the
values are input and how the values are verified.  The load and verify secrets
are isolated to their own access programs.

\subsection{Module}

Param

\subsection{Uses}

SpecParam (Section~\ref{SpecParam})

\subsection{Syntax}

\begin{tabular}{p{3cm} p{1cm} p{1cm} >{\raggedright\arraybackslash}p{9cm}}
\toprule
\textbf{Name} & \textbf{In} & \textbf{Out} & \textbf{Exceptions} \\
\midrule
load\_params & string & - &  FileTypeError \\
verify\_params & - & - & badLength, badDiam, outofboundary, negativevalue,
                        badTotalPower, badRenewableRate, badGridPowerRate,
                        badDistances\\
$d_i$ & - & $\mathbb{R}$\\
$C_R$ & - & $\mathbb{R}$\\
$C_P$ & - & $\mathbb{R}$\\
$L_T$ & - & $\mathbb{R}$\\
... & ... & ...\\
$L_i$ & - & $\mathbb{R}$ \\
\bottomrule
\end{tabular}

\subsection{Semantics}

\subsubsection{Environment Variables}

inputFile: sequence of string \#\textit{f[i] is the ith string in the text file f}\\ 

\subsubsection{State Variables}

\renewcommand{\arraystretch}{1.2}
\begin{longtable*}[l]{l} 
\# From T1\\
$R_i$: $\mathbb{R}$ \\
$P_i$: $\mathbb{R}$ \\
$C_T$: $\mathbb{R}$ \\
$C_r$: $\mathbb{R}$ \\
$C_p$ : $\mathbb{R}$ \\
~\\
\# From T2\\
$T_L$: $\mathbb{R}$ \\
$P_i$: $\mathbb{R}$ \\
$d_i$: $\mathbb{R}$ \\
~\\
\noindent \# From T3\\
$L_T$: $\mathbb{R}$ \\
$L_i$: $\mathbb{R}$ \\
$R_i$: $\mathbb{R}$ \\
$P_i$: $\mathbb{R}$ \\
~\\
\# To Support IM1\\
$L_T$: $\mathbb{R}$ \\
$L_i$: $\mathbb{R}$ \\ 
~\\

\end{longtable*}

\subsubsection{Assumptions}

\begin{itemize}

\item readtable(filename) will be called before the values of any state variables will be accessed.

\item The file contains the string equivalents of the numeric values for
each input parameter in order, each on a new line. The order is the same as in
the table in R1 of the SRS. Any comments in the input file should be denoted
with a '\#' symbol.

\end{itemize}

\subsubsection{Access Routine Semantics}

\noindent Param.$R_i$:
\begin{itemize}
\item output: \textit{out} := $R_i$
\item exception: none
\end{itemize}

\noindent Param.$P_i$:
\begin{itemize}
\item output: \textit{out} := $P_i$
\item exception: none
\end{itemize}

...
~\newline

\noindent Param.$T_L$:
\begin{itemize}
\item output: \textit{out} := $T_L$
\item exception: none
\end{itemize}

\noindent Param.$L_i$:
\begin{itemize}
\item output: \textit{out} := $L_i$
\item exception: none
\end{itemize}
\noindent Param.$C_T$:
\begin{itemize}
\item output: \textit{out} := $C_T$
\item exception: none
\end{itemize}
\noindent load\_params($s$):

\noindent Param.$C_r$:
\begin{itemize}
\item output: \textit{out} := $C_r$
\item exception: none
\end{itemize}
\noindent Param.$C_p$:
\begin{itemize}
\item output: \textit{out} := $C_p$
\item exception: none
\end{itemize}
\noindent Param.$L_T$:
\begin{itemize}
\item output: \textit{out} := $L_T$
\item exception: none
\end{itemize}
\noindent Param.$L_max$:
\begin{itemize}
\item output: \textit{out} := $L_\text{max}$
\item exception: none
\end{itemize}
\noindent Param.$d_i$:
\begin{itemize}
\item output: \textit{out} := $d_i$
\item exception: none
\end{itemize}


\noindent \begin{longtable*}[l]{l l} 
$\neg (C_r < 0)$ & $\Rightarrow$ badValue\\
$\neg (d_i > 2000)$ & $\Rightarrow$ warnLength\\
$\neg (C_p < 0)$ & $\Rightarrow$ badValue\\
$\neg (0 \leq L_T \leq L_\text{max})$ & $\Rightarrow$ warnLength\\
$\neg (d_i < 0)$ & $\Rightarrow$ badValue\\
$\neg (L_T < 0)$ & $\Rightarrow$ badValue\\
$\neg (L_\text{max} < 0)$ & $\Rightarrow$ badValue\\
\end{longtable*}

etc.  See Appendix (Section~\ref{Appendix}) for the complete list of exceptions and
associated error messages.

\subsection{Considerations}

The value of each state variable can be accessed through its name (getter).  An
access program is available for each state variable.  There are no setters for
the state variables, since the values will be set and checked by load params and
not changed for the life of the program.

\newpage

 \section{MIS of Input Verification Module} \label{VerifyInput}

 \subsection{Module}

 verify\_params

 \subsection{Uses}

 Param (Section~\ref{Parameters})

 \subsection{Syntax}

 \subsubsection{Exported Access Programs}

 \begin{center}
 \begin{tabular}{p{3cm} p{1cm} p{1cm} p{9cm}}
 \hline
 \textbf{Name} & \textbf{In} & \textbf{Out} & \textbf{Exceptions} \\
 \hline
 verify\_valid & - & - & badLength, badDiam, outofboundary, negativevalue,
                        badTotalPower, badRenewableRate, badGridPowerRate,
                        badDistances\\
 \hline
 verify\_recommend & - & - & - \\
 \hline
 \end{tabular}
 \end{center}

 \subsection{Semantics}

\subsubsection{Environment Variables}
Distances upper boundary

 \subsubsection{Assumptions}

All of the fields Param have been assigned values before any of the access
 routines for this module are called.

 \subsubsection{Access Routine Semantics}

 verify\_valid(): 
 \begin{itemize}
 \item transition: none
 \item exceptions: exc := (\\
 Param.get$d_i$() $\leq 0 \Rightarrow$ badLength $|$\\
 Params.get$C_r$() $\leq 0 \Rightarrow$ badValue $|$\\
 Params.get$L_T$() $\leq$ $L_\text{max}$ $\Rightarrow$ badValue$|$\\
 Params.get$C_p$() $\leq 0 \Rightarrow$ badValue $|$\\
 Params.get$L_\text{max}$() $\leq 0 \Rightarrow$ badValue $|$\\ 
 \end{itemize}

 
 \subsection{Considerations}

See Appendix (Section~\ref{Appendix}) for the complete list of exceptions and
 associated error messages.

\newpage
\section{MIS of Optimization Module} \label{Temperature}

\subsection{Module}

Minimize total cost

\subsection{Uses}

Param (Section~\ref{Parameters})

\subsection{Syntax}

\subsubsection{Exported Access Programs}

\begin{center}
\begin{tabular}{p{3.5cm} p{1cm} p{7cm} p{2cm}}
\hline
\textbf{Name} & \textbf{In} & \textbf{Out} & \textbf{Exceptions} \\
\hline
linprog & -- & $(\mathbb{R} \rightarrow \mathbb{R})$ & - \\
\hline
optimoptions& -- &  $(\mathbb{R}\rightarrow \mathbb{R})$ & - \\
\hline
transloss& -- &  $(\mathbb{R}\rightarrow \mathbb{R})$ & - \\
\hline
\end{tabular}
\end{center}

\subsection{Semantics}

\subsubsection{State Variables}

none

\subsubsection{Assumptions}

none

\subsubsection{Access Routine Semantics}

linprog(): 
\renewcommand*{\arraystretch}{1.5}
\begin{itemize}
\item output:
$A_{eq} \cdot x = b_{eq}$ := \text{solve}(\text{linp\_optimization}, $L_T$,
$P_T$, $R_T$, $C_T$, $C_R$, $C_G$,)\\
\item exception: none

\end{itemize}

optimoptions(): 
\renewcommand*{\arraystretch}{1.5}
\begin{itemize}
\item output: $L_T$ := solve(linp\_optimization,
$R_i^N$, $P_i^N$)\\
\item exception: none
\end{itemize}
transloss(): 
\renewcommand*{\arraystretch}{1.5}
\begin{itemize}
\item output:
$T_L = (d_i \cdot 0.03 \cdot P_i)^{N}$\\
\item exception: none
\end{itemize}
\newpage


\section{MIS of Output Verification Module} \label{VerifyOutput}

\subsection{Module}

verify\_output

\subsection{Uses}

Param (Section~\ref{Parameters})

\subsection{Syntax}

\subsubsection{Exported Constant}

None



\subsection{Semantics}

\subsubsection{State Variables}

None

\subsubsection{Assumptions}

All of the fields of the input parameters structure have been assigned a
value.  

\subsubsection{Access Routine Semantics}

\noindent verify\_output($L_i$, $d_i$, $R_i$, $P_i$, $C_T$, $L_T$, $L_\text{max}$):
\begin{itemize}
\item verification := ($L_T$ =$\sum_{i=1}^{n} $L_i$)$
\end{itemize}


\newpage
\section{MIS of Plotting Module} \label{Plot}

\subsection{Module}

plot

\subsection{Uses}

N/A

\subsection{Syntax}

\subsubsection{Exported Access Programs}

\begin{center}
\begin{tabular}{p{2cm} p{8cm} p{2cm} p{2cm}}
\hline
\textbf{Name} & \textbf{In} & \textbf{Out} & \textbf{Exceptions} \\
\hline
plot & $d_i:\mathbb{R} \rightarrow \mathbb{R},
                 L_i:\mathbb{R} \rightarrow \mathbb{R},
                 $ & - & - \\
\hline
\end{tabular}
\end{center}

\subsection{Semantics}

\subsubsection{State Variables}

None

\subsubsection{Environment Variables}

will display on MATLAB within it own graph\\

\subsubsection{Assumptions}

None

\subsubsection{Access Routine Semantics}

\noindent plot($d_i$, $L_i$):
\begin{itemize}
\item transition: To display a plot where the vertical axis
  The power consumption distribution and the horizontal axis is the distance between data centers and power stations.  
\item exception: none
\end{itemize}

\newpage

\section{MIS of Output Module} \label{Output}

\subsection{Module}

output

\subsection{Uses}

Param (Section~\ref{Parameters})

\subsection{Syntax}

\subsubsection{Exported Constants}

totalcost: integer

\subsubsection{Exported Access Program}

\begin{center}
\begin{tabular}{p{3cm} p{7cm} p{2cm} p{2cm}}
\hline
\textbf{Name} & \textbf{In} & \textbf{Out} & \textbf{Exceptions} \\
\hline
output & filename: string, $C_T:\mathbb{R} \rightarrow \mathbb{R},
                 L_i:\mathbb{R} \rightarrow \mathbb{R},
                 P_i:\mathbb{R} \rightarrow \mathbb{R},
       R_i:\mathbb{R} \rightarrow \mathbb{R}$, $L_T: \mathbb{R}$ & - & - \\
\hline
\end{tabular}
\end{center}

\subsection{Semantics}

\subsubsection{State Variables}

None

\subsubsection{Environment Variables}

file: A text file

\subsubsection{Access Routine Semantics}

\noindent output(filename, $C_T$, $L_i$, $L_T$, $R_i$, $P_i$):
\begin{itemize}
\item transition:  Write and export the result into a file the
  following: the input
    parameters from Param, and the calculated values $C_T, L_i, R_i,
    P_i$.  The functions will be output as
    sequences in this file.  The spacing between points in the sequence should
    be selected so that the heating behaviour is captured in the data.
\item exception: none
\end{itemize}

\newpage

\section{MIS of Specification Parameters} \label{SpecParam}

The secrets of this module is the value of the specification parameters.

\subsection{Module}

SpecParam

\subsection{Uses}

N/A

\subsection{Syntax}

\subsubsection{Exported Constants}

\renewcommand{\arraystretch}{1.2}
\begin{longtable*}[l]{l} 
\# Some Default Value\\
  $L_\text{max}$ := 6\\
  $L_T$ := 10\\
  $C_r$ := 0.041 \\
  $C_p$ := 0.009 \\
  $N$ := 5\\
  $d_1$ := 10 \\
  $d_2$ := 20\\
  $d_3$ := 30\\
  $d_4$ := 40 \\
  $d_5$ := 50\\
  
\end{longtable*}



\subsection{Semantics}

N/A

\newpage

\bibliographystyle {plainnat}
\bibliography {MIS}

\newpage

\section{Appendix} \label{Appendix}

\renewcommand{\arraystretch}{1.2}

\begin{longtable}{l p{12cm}}
\caption{Possible Exceptions} \\
\toprule
\textbf{Message ID} & \textbf{Error Message} \\
\midrule
badValue & Error: Input Value must be $> 0$ \\
linprog & Error: no feasible region \\
Error: Wrong input type \\

\bottomrule
\end{longtable}


\end{document}
