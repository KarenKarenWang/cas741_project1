% THIS DOCUMENT IS TAILORED TO REQUIREMENTS FOR SCIENTIFIC COMPUTING.  IT SHOULDN'T
% BE USED FOR NON-SCIENTIFIC COMPUTING PROJECTS
\documentclass[12pt]{article}

\usepackage{amsmath, mathtools}
\usepackage{amsfonts}
\usepackage{amssymb}
\usepackage{graphicx}
\usepackage{colortbl}
\usepackage{xr}
\usepackage{hyperref}
\usepackage{longtable}
\usepackage{xfrac}
\usepackage{tabularx}
\usepackage{float}
\usepackage{siunitx}
\usepackage{booktabs}
\usepackage{caption}
\usepackage{pdflscape}
\usepackage{afterpage}

\usepackage[round]{natbib}

%\usepackage{refcheck}

\hypersetup{
    bookmarks=true,         % show bookmarks bar?
      colorlinks=true,       % false: boxed links; true: colored links
    linkcolor=red,          % color of internal links (change box color with linkbordercolor)
    citecolor=green,        % color of links to bibliography
    filecolor=magenta,      % color of file links
    urlcolor=cyan           % color of external links
}

\input{./Comments}
%% Common Parts

\newcommand{\progname}{Time\_Freq\_Analysis} % PUT YOUR PROGRAM NAME HERE %Every program
                                % should have a name


% For easy change of table widths
\newcommand{\colZwidth}{1.0\textwidth}
\newcommand{\colAwidth}{0.13\textwidth}
\newcommand{\colBwidth}{0.82\textwidth}
\newcommand{\colCwidth}{0.1\textwidth}
\newcommand{\colDwidth}{0.05\textwidth}
\newcommand{\colEwidth}{0.8\textwidth}
\newcommand{\colFwidth}{0.17\textwidth}
\newcommand{\colGwidth}{0.5\textwidth}
\newcommand{\colHwidth}{0.28\textwidth}

% Used so that cross-references have a meaningful prefix
\newcounter{defnum} %Definition Number
\newcommand{\dthedefnum}{GD\thedefnum}
\newcommand{\dref}[1]{GD\ref{#1}}
\newcounter{datadefnum} %Datadefinition Number
\newcommand{\ddthedatadefnum}{DD\thedatadefnum}
\newcommand{\ddref}[1]{DD\ref{#1}}
\newcounter{theorynum} %Theory Number
\newcommand{\tthetheorynum}{T\thetheorynum}
\newcommand{\tref}[1]{T\ref{#1}}
\newcounter{tablenum} %Table Number
\newcommand{\tbthetablenum}{T\thetablenum}
\newcommand{\tbref}[1]{TB\ref{#1}}
\newcounter{assumpnum} %Assumption Number
\newcommand{\atheassumpnum}{P\theassumpnum}
\newcommand{\aref}[1]{A\ref{#1}}
\newcounter{goalnum} %Goal Number
\newcommand{\gthegoalnum}{P\thegoalnum}
\newcommand{\gsref}[1]{GS\ref{#1}}
\newcounter{instnum} %Instance Number
\newcommand{\itheinstnum}{IM\theinstnum}
\newcommand{\iref}[1]{IM\ref{#1}}
\newcounter{reqnum} %Requirement Number
\newcommand{\rthereqnum}{P\thereqnum}
\newcommand{\rref}[1]{R\ref{#1}}
\newcounter{nfrnum} %NFR Number
\newcommand{\rthenfrnum}{NFR\thenfrnum}
\newcommand{\nfrref}[1]{NFR\ref{#1}}
\newcounter{lcnum} %Likely change number
\newcommand{\lthelcnum}{LC\thelcnum}
\newcommand{\lcref}[1]{LC\ref{#1}}

\usepackage{fullpage}

\newcommand{\deftheory}[9][Not Applicable]
{
\newpage
\noindent \rule{\textwidth}{0.5mm}

\paragraph{RefName: } \textbf{#2} \phantomsection 
\label{#2}

\paragraph{Label:} #3

\noindent \rule{\textwidth}{0.5mm}

\paragraph{Equation:}

#4

\paragraph{Description:}

#5

\paragraph{Notes:}

#6

\paragraph{Source:}

#7

\paragraph{Ref.\ By:}

#8

\paragraph{Preconditions for \hyperref[#2]{#2}:}
\label{#2_precond}

#9

\paragraph{Derivation for \hyperref[#2]{#2}:}
\label{#2_deriv}

#1

\noindent \rule{\textwidth}{0.5mm}

}

\begin{document}

\title{Software Requirements Specification for \progname:
A program for distributing computational power for data centers} 
\author{\authname}
\date{\today}
	
\maketitle

~\newpage

\pagenumbering{roman}

\tableofcontents

~\newpage

\section*{Revision History}

\begin{tabularx}{\textwidth}{p{3cm}p{2cm}X}
\toprule {\bf Date} & {\bf Version} & {\bf Notes}\\
\midrule
\today & 1.0 & Initial Release\\
\bottomrule
\end{tabularx}

~\newpage

\section{Reference Material}

This section records information for easy reference.

\subsection{Table of Units}

Throughout this document SI (Syst\`{e}me International d'Unit\'{e}s) is employed
as the unit system.  In addition to the basic units, several derived units are
used as described below.  For each unit, the symbol is given followed by a
description of the unit and the SI name.
~\newline

\renewcommand{\arraystretch}{1.2}
%\begin{table}[ht]
  \noindent \begin{tabular}{l l l} 
    \toprule		
    \textbf{symbol} & \textbf{unit} & \textbf{SI}\\
    \midrule 
    \si{\metre} & length & metre\\
    \si{\kWh} & energy	& kilowatt\\
    \si{\m^3} & volume	& cubic meter\\
    \si{\second} & time & second\\
    \si{\celsius} & temperature & centigrade\\
    \si{HM/s} & computational power & hashrate\\
    \si{\watt} & power & watt (W = \si{\joule\per\second})\\
    \bottomrule
  \end{tabular}
  %	\caption{Provide a caption}
%\end{table}



Additionally, currency for the total cost, is labelled by unit "CAD" which define Canadian dollars.

%{The symbol for units named after people use capital letters, but the name of the unit itself uses lower case.  For instance, pascals use the symbol Pa, watts use the symbol W, teslas use the symbol T, newtons use the symbol N, etc.  The one exception to this is degree Celsius.  Details on writing metric units can be found on the 
 % \href{https://www.nist.gov/pml/weights-and-measures/writing-metric-units}
 % {NIST} web-page.}

\subsection{Table of Symbols}

The table that follows summarizes the symbols used in this document along with their units. The choice of symbols was made to be consistent with existing documentation for electricity distribution systems.  The symbols are listed in alphabetical order.

\renewcommand{\arraystretch}{1.2}
%\noindent \begin{tabularx}{1.0\textwidth}{l l X}
\noindent \begin{longtable*}{l l p{12cm}} \toprule
\textbf{symbol} & \textbf{unit} & \textbf{description}\\
\midrule 
$R_\text{total}$ & \si[per-mode=symbol] {\cubic\metre} &carbon dioxide emission\\
$L_\text{total}$ & \si[per-mode=symbol] {\watt} &total power demanded by the given combination plan
\\ 
$R_\text{total}$ & \si[per-mode=symbol] {\watt} &total electricity supply by the renewable energy\\
$P_\text{total}$ & \si[per-mode=symbol] {\watt} &total electricity supply by the distributed energy\\
\bottomrule
\end{longtable*}
%{Use your problems actual symbols.  The si package is a good idea to use for units.}

\subsection{Abbreviations and Acronyms}

\renewcommand{\arraystretch}{1.2}
\begin{tabular}{l l} 
  \toprule		
  \textbf{symbol} & \textbf{description}\\
  \midrule 
  Ctotal & total cost\\
  Cr & renewable cost\\
  Cp & grid electricity cost\\
  di & Distance between data center and power station\\
  Ai & Computational power for data center\\
  IM & Instance Model\\
  LC & Likely Change\\
  PS & Physical System Description\\
  R & Requirement\\
  SRS & Software Requirements Specification\\
%  \progname{} & {put an expanded version of your program name here (as appropriate)}\\

  \bottomrule
\end{tabular}\\

%{Add any other abbreviations or acronyms that you add}

%\subsection{Mathematical Notation}

%{This section is optional but should be included for projects that make use of notation to convey mathematical information.  For instance, if typographic}
 

%{This section was added to the template because some students use very.}


\newpage
\section{Introduction}

 The COVID-19 pandemic has clearly shown that rural hospitals are a weak link in regional healthcare systems. They lack the advanced digital services, human and financial resources, and integration with urban hospitals that are needed to provide excellent health care uniformly across a provincial region. Data and Computing Facilities (DCFs) are established in rural hospitals. To meet the challenge of COVID-19 and other public health crises, this model was developed aiming at solving part of the rural hospital's problems. Data centers use about 1\%-4\% of the electricity in the world\textsuperscript{\cite{rivier2013electricity}}. Using energy more efficiently is one of the fastest, most cost-effective ways to save money, reduce greenhouse gas emissions, create jobs, and meet growing energy demand.
%First, Data/Computing Facilities (DCFs) are established in rural hospitals. Built from modular, manufacturable units; housed in underutilized legacy space; and efficiently powered using a multi-energy model, each DCF operates as a digital ecosystem.
\subsection{Purpose of Document}

{The purpose of this document is to provide a analytical model of allocate the computing power to each data center with minimize the cost to intended Reader (Section~\ref{sec_assumpt}).  This document should clearly
communicate all necessary background information and detailed system context is described in Section 3.}

\subsection{Scope of Requirements} 

 {The domain of this problem is restricted to given renewable pricing and factory electricity will lose proportional over transmission . In particular, the different areas will have different electricity pricing and different types of renewable resources. However, we can set the constant renewable pricing and minimize the total cost to the final the optimal computing distributed plan for each data center. The domain to one specific type of input data which is distance between data center and the power station in this program, specifically, we will find out the optimization cases of combination under static pricing. Therefore, it has been decided to scope distances as input. Subsequently, since this project is limited to any renewable energy with a given price, any transmission power loss of renewable energy is considered out of scope. Essentially, while this program is intended to eventually be used within the context of specific physical data, it is developed and constructed for application to another area in North America as well.}  

%{The scope section is related to the assumptions section(Section~\ref{sec_assumpt}).  However, the scope and the assumptions are not at the same level of abstraction.  The scope is at a high level.  The focus is on the ``big picture'' assumptions.  The assumptions section lists, and describes, all of the assumptions.}

\subsection{Characteristics of Intended Reader} \label{sec_IntendedReader}

{This document assumes the intended reader has familiarity with basic optimization analysis, power system analysis, and electricity distribution analysis. Courses which contribute to background knowledge may be titled Linear Systems (undergrad), Electric Machines and Power Systems (undergrad), Electric Power Transmission, Distribution and Utilization (undergrad), Sustainable Electrical Power Systems (Graduate), and Real Analysis (Graduate). Data constrain and model limitation will be discussed in this document. However, our exposition will only cover the concepts needed for our purposes. For proofs and for a complete exposition of all background materials. Furthermore, this document will explain as entry as possible aiming to appliable for more readers.}


\subsection{Organization of Document}

{This document is built on the template recommendations in cas 741 gitlab that seeks to standardize communication tools for software development. The suggested order for reading this SRS document is: Goal Statement, Instance Models, Requirements, Introduction, and Specific System Description.}

\section{General System Description}

{This section provides general information about the system.  It identifies the interfaces between the system and its environment, describes the user
characteristics and lists the system constraints.}

\subsection{System Context}

{The following figure depicts a system context view of the computing relocation model.}

\begin{figure}[h!]
\begin{center}
 \includegraphics[width=0.6\textwidth]{system context Figure.png}
\caption{System Context}
\label{Fig_SystemContext} 
\end{center}
\end{figure}

{For each of the entities in the system context diagram its responsibilities
  should be listed.  Whenever possible the system should check for data quality,
  but for some cases the user will need to assume that responsibility.  The list
  of responsibilites should be about the inputs and outputs only, and they
  should be abstract.  Details should not be presented here.  However, the
  information should not be so abstract as to just say ``inputs'' and
  ``outputs''.  A summarizing phrase can be used to characterize the inputs.
  For instance, saying ``material properties'' provides some information, but it
  stays away from the detail of listing every required property.}

\begin{itemize}
\item User Responsibilities:
\begin{itemize}
\item keep the unit even, keep every input with SI standard
\item Provide the distances between the power station and data centers and keep input under constraint.
\item Change system setting with their unique situations included total power consumption, renewable power rate and grid power rate.

\end{itemize}
\item \progname{Minimization Analysis} Responsibilities:
\begin{itemize}
\item Keep positive input and less than max limit
\item Detect if the output is capable for a data center to carry
\item Detect data type mismatch, such as relocate data under same unit
\end{itemize}
\end{itemize}

\subsection{User Characteristics} \label{SecUserCharacteristics}

{The front user should keep the unit with SI standard. The end user should have basic knowledge of undergraduate math and physics.}

\subsection{System Constraints}

{The method developed in this project is expected to be independent of system constraints at this time.}

\section{Specific System Description}

This section first presents the problem description, which gives a high-level
view of the problem to be solved.  This is followed by the solution characteristics
specification, which presents the assumptions, theories, definitions and finally
the instance models.  {Add any project-specific details that are relevant
  for the section overview.}

\subsection{Problem Description} \label{Sec_pd}

\progname{This model is intended to estimate the minimum cost with different computing distributions of data centers with different distances from electricity power station} \\

\subsubsection{Terminology and  Definitions}\label{ssc:terminology-definitions}
\label{ssc:TM}

 {Minimization problem: The function to minimize is called the objective function. The minimum value of the objective function is in the margins of the feasible area delimited by the restrictions of the problem. This value is called the ideal value.
   }



\begin{itemize}
\item Construct a graph for feasible region
\item find a feasible region and get the optimized combinations of computing distributions

\end{itemize}

\subsubsection{Physical System Description} \label{sec_phySystDescrip}

{Physics system will regarding to real world. HVDC transmission losses are quoted as less than 3\% per 1,000 km\textsuperscript{\cite{rosellon2003different}}, a formula was designed to calculate the total power lose along the transmission line. With various of distance as input, the system operation like a net flowing. Each data center will consuming electricity from both renewable resources and grid power station. In the model, only transmission along electricity grid will assume have power lose, renewable resource stations are assuming very close to the data centers the supplied.  }

The physical system of \progname{}, as shown in Figure2,
includes the following elements:

\begin{itemize}

\item Data centers are not uniformed in the area.

\item Renewable resources stations are very close to the data centers they supplied.

\end{itemize}
\begin{figure}[h!]
\begin{center}
 \includegraphics[width=0.6\textwidth]{distributed model.pdf}
\caption{System Context}
\label{Fig_SystemContext} 
\end{center}
\end{figure}


\subsubsection{Goal Statements}

Give a bunch of distances between data centers and electric power station as input, total power consumption was set, solving minimization problems: 

%\noindent Given the {inputs}, the goal statements are:

\begin{itemize}

\item[GS\refstepcounter{goalnum}\thegoalnum \label{G_meaningfulLabel}:] {This model is intended to estimate the different computing distributions of data centers.}
\item[GS\refstepcounter{goalnum}\thegoalnum \label{G_meaningfulLabel}:] {This model is intended to output final solution under the minimum cost.}
\item[GS\refstepcounter{goalnum}\thegoalnum \label{G_meaningfulLabel}:] {This model possible to output the CO2 emission for reference.}
\end{itemize}

\subsection{Solution Characteristics Specification}

 {This section characterizes the attributes of an acceptable solution. Both analysts and stakeholders should agree on these attributes so that the solution can be accepted when the project is complete.}


\subsubsection{Assumptions} \label{sec_assumpt}


This section simplifies the original problem and helps in developing the
theoretical model by filling in the missing information for the physical
system. The numbers given in the square brackets refer to the theoretical model
[T], general definition [GD], data definition [DD], instance model [IM], or
likely change [LC], in which the respective assumption is used.
\begin{itemize}

\item[A\refstepcounter{assumpnum}\theassumpnum \label{as:n}:]
  We know that distances as input should be positive and under reachable constrain.

\item[A\refstepcounter{assumpnum}\theassumpnum \label{as:approximate}:]
  The software will estimate the distribution mof computing power from a constant number relocated. 

\item[A\refstepcounter{assumpnum}\theassumpnum \label{as:threeterm}:]
  The scope of minimization analysis is to compute the optimal plan that also most cost efficiency.

\item[A\refstepcounter{assumpnum}\theassumpnum \label{as:sixterm}:]
  The scope of set power lose with a constant coefficient is to get rid of substations power lose effect.
\item[A\refstepcounter{assumpnum}\theassumpnum \label{A_meaningfulLabel}:]
  {Depends on the pricing with $C_r$ and $C_p$,  the $L_i$ will distribute with descending or ascending order of di }

\end{itemize}

\subsubsection{Theoretical Models}\label{TM-series-cauchy-condition}\\
Applying the terminology and definitions from 4.1.1, this section records
theorems required to identify a minimum optimal solution.

Consider the total cost for all data centers this is the sum of the total grid power supply cost plus the total renewable supply cost. The computing power distribution will as the same proportion as their electricity consumption\textsuperscript{\cite{sedano2004electricity}}.
~\newline

\noindent
\deftheory
% #2 refname of theory
{T1}\label{TM-series-cauchy-condition}\\
% #3 label
{Minimize the cost}
% #4 equation
{
  $C_\text {total}$=\sum_{i=1}^{n} C r i * R i+C P i * P i
}
% #5 description
{
  The above equation gives the value of the total cost for given Cr and Cp with power consumption as Ri and Pi respectively.
  When solving a minimizing problem, some constrain of variables must apply to find the feasible region. 
}
% #6 Notes
{
None.
}
% #7 Source
{
  \url{https://www.superprof.co.uk/resources/academic/maths/linear-algebra/linear-programming/steps-to-solve-a-linear-programming-problem.html}
}
% #8 Referenced by
{
  GD2
}
% #9 Preconditions
{
None
}
% #1 derivation - not applicable by default
{}
\noindent
\deftheory
% #2 refname of theory
{T2}
% #3 label
{Calculate for Pi}
% #4 equation
{
  \mathrm{Pi}=\mathrm{Pi-theoretical} /(1-(0.03 \mathrm{di} / 1000)) 
}
% #5 description
{
  The above equation gives the value of power lose with transmission distance varied. The Pi is the actual consumption of energy, Pi-theoretical is the consumption without transmission lose. HVDC transmission losses are quoted as less than 3\% per 1,000 km 
}
% #6 Notes
{
None.
}
% #7 Source
{
  \url{https://www.superprof.co.uk/resources/academic/maths/linear-algebra/linear-programming/steps-to-solve-a-linear-programming-problem.html}
}
% #8 Referenced by
{
  GD1
}
% #9 Preconditions
{
None
}
\noindent
\deftheory
% #2 refname of theory
{T3}
% #3 label
{The power consumption for each data center}
% #4 equation
{
  $L_\text {total}$=\sum_{i=1}^{n} $L_i$ = $R_i$ + $P_i$
}
% #5 description
{
  The power consumption for each data center will come up with renewable supply and grid power supply. 
}
% #6 Notes
{
None.
}
% #7 Source
{
  \url{https://www.superprof.co.uk/resources/academic/maths/linear-algebra/linear-programming/steps-to-solve-a-linear-programming-problem.html}
}
% #8 Referenced by
{
 GD1
}
% #9 Preconditions
{
None
}
{``Ref.\ By'' is used repeatedly with the different types of information. This stands for Referenced By.  It means that the models, definitions and
  assumptions listed reference the current model, definition or assumption.
  This information is given for traceability.  Ref. By provides a pointer in the
  opposite direction to what we commonly do.  You still need to have a reference
  in the other direction pointing to the current model, definition or
  assumption.  As an example, if T1 is referenced by G2, that means that G2 will
  explicitly include a reference to T1.}

~\newline

\subsubsection{General Definitions}\label{sec_gendef}

General Definitions (GDs) are a refinement of one or more TMs, and/or of
  other GDs.  The GDs are less abstract than the TMs.  Generally the reduction
  in abstraction is possible through invoking referencing Assumptions. This section collects the laws and equations that will be used in building the
instance models.

~\newline

\noindent
\begin{minipage}{\textwidth}
\renewcommand*{\arraystretch}{1.5}
\begin{tabular}{| p{\colAwidth} | p{\colBwidth}|}
\hline
\rowcolor[gray]{0.9}
Number& GD\refstepcounter{defnum}\thedefnum \label{NL}\\
\hline
Label &\bf Power losses along transmission line \\
\hline
% Units&$MLt^{-3}T^0$\\
% \hline
SI Units&\si{\kWh}\\
\hline
Equation& see description  \\
\hline
Description &  
                Transmission and distribution (T\&D) loss are amounts that are not paid for by users.

                T\&D Losses= (Energy Input to feeder (Kwh)-Billed Energy to Consumer (Kwh)) / Energy Input kwh x100.
\\

\\
\hline
  Source & \url{https://www.electricalindia.in/losses-in-distribution-transmission-lines/#:~:text=Transmission%20and%20distribution%20(T%26D,))%20%2F%20Energy%20Input%20kwh%20x100.} \\
  \hline
  Ref.\ By & T2\\
  \hline
\end{tabular}
\end{minipage}\\

\subsubsection*{Detailed derivation of calculating power transmission losses}

{It is fact that the unit of electric energy generated by the power station does not match the units distributed to the consumers. Some percentage of the units is lost in the distribution network. This difference in the generated \& distributed units is known as transmission and distribution losses. Distribution Sector is considered the weakest link in the entire power sector. Transmission losses are approximately 17\% while distribution losses are approximately 50\%. In this model, we assume all of the data centers pass through the same number of sectors, so only transmission line losses are calculated.}

\subsubsection{Data Definitions}\label{sec_datadef}

{The Data Definitions are definitions of symbols and equations that are
  given for the problem.  They are not derived; they are simply used by other
  models.  For instance, if a problem depends on density, there may be a data
  definition for the equation defining density.  The DDs are given information
  that you can use in your other modules.}

{All Data Definitions should be used (referenced) by at least one other
  model.}

This section collects and defines all the data needed to build the instance
models. The dimension of each quantity is also given.  {Modify the examples
  below for your problem, and add additional definitions as appropriate.}

~\newline

\noindent
\begin{minipage}{\textwidth}
\renewcommand*{\arraystretch}{1.5}
\begin{tabular}{| p{\colAwidth} | p{\colBwidth}|}
\hline
\rowcolor[gray]{0.9}
Number& DD\refstepcounter{datadefnum}\thedatadefnum \label{FluxCoil}\\
\hline
Label& \bf Level of computing power for each data center\\
\hline
Symbol &$A_p$\\
\hline
% Units& $Mt^{-3}$\\
% \hline
  SI Units & \si{MH/s}\\
  \hline
  Equation& None\\
  \hline
  Description & 
               Hashrate is a measure of the computational power per second used when mining. More simply, it is the speed of mining. It is measured in units of hash/second, meaning how many calculations per second can be performed. Machines with a high hash power are highly efficient and can process a lot of data in a single second. In the case of Bitcoin, the hashrate indicates the number of times hash values are calculated for PoW every second.

                Hashrate is usually measured in units of k (kilo, 1,000), M (mega, 1 million), G (giga, 1 billion), or T (tera, 1 trillion). For example, 1 Mhash/s indicates 1 million hash calculations are done every second.
  \\
  \hline
  Sources& \url{https://bitflyer.com/en-eu/s/glossary/hashrate}\\
  \hline
  Ref.\ By & \iref{ewat}\\
  \hline
\end{tabular}
\end{minipage}\\


\subsubsection{Instance Models} \label{sec_instance}    

{The motivation for this section is to reduce the problem defined in
  ``Physical System Description'' (Section~\ref{sec_phySystDescrip}) to one
  expressed in mathematical terms. The IMs are built by refining the TMs and/or
  GDs.  This section should remain abstract.  The SRS should specify the
  requirements without considering the implementation.}

This section transforms the problem defined in Section~\ref{Sec_pd} into 
one which is expressed in mathematical terms. It uses concrete symbols defined 
in Section~\ref{sec_datadef} to replace the abstract symbols in the models 
identified in Sections~\ref{sec_theoretical} and~\ref{sec_gendef}.

The goals {reference your goals} are solved by {reference your instance
  models}.  {other details, with cross-references where appropriate.}
{Modify the examples below for your problem and add additional models as
  appropriate.}

~\newline

%Instance Model 1

\noindent
\begin{minipage}{\textwidth}
\renewcommand*{\arraystretch}{1.5}
\begin{tabular}{| p{\colAwidth} | p{\colBwidth}|}
  \hline
  \rowcolor[gray]{0.9}
  Number& IM\refstepcounter{instnum}\theinstnum \label{ewat}\\
  \hline
  Label& \bf electricity consumption distribution to find Computational power distribution\\
  \hline
  Input&$L_i$, $A_\text{total}$, $L_\text{total}$\\
  & The input is constrained so that all variables are non-negative\\
  \hline
  Output&$A_1$, $A_2$...$A_n$\\
  &$L_\text{total}$/$L_i$=$A_\text{total}$/$A_i$ \\
  \hline
  Description& $L_i$is the power consumption by each data center.\\
  &$L_\text{total}$ is the total power consumption for data centers in this model.\\
  &$A_i$ is the Computational power for each data center.\\
  &$A_\text{total}$ is the total Computational power for data center in this model \\

  \hline
  Sources& \url{https://www.iea.org/reports/data-centres-and-data-transmission-networks} \\
  \hline
  Ref.\ By & DD1\\
  \hline
\end{tabular}
\end{minipage}\\

%~\newline


\subsubsection{Input Data Constraints} \label{sec_DataConstraints}    

Table~\ref{TblInputVar} shows the data constraints on the input output
variables.  The column for physical constraints gives the physical limitations
on the range of values that can be taken by the variable.  The column for
software constraints restricts the range of inputs to reasonable values.  The
software constraints will be helpful in the design stage for picking suitable
algorithms.  The constraints are conservative, to give the user of the model the
flexibility to experiment with unusual situations.  The column of typical values
is intended to provide a feel for a common scenario.  The uncertainty column
provides an estimate of the confidence with which the physical quantities can be
measured.  This information would be part of the input if one were performing an
uncertainty quantification exercise.

The specification parameters in Table~\ref{TblInputVar} are listed in
Table~\ref{TblSpecParams}.

\begin{table}[!h]
  \caption{Input Variables} \label{TblInputVar}
  \renewcommand{\arraystretch}{1.2}
\noindent \begin{longtable*}{l l l l c} 
  \toprule
  \textbf{Var} & \textbf{Physical Constraints} & \textbf{Software Constraints} &
                             \textbf{Typical Value} & \textbf{Uncertainty}\\
  \midrule 
  $di$ & $di > 0$ & 0 \leq di \leq $d_{\text{max}}$ & 1000 \si[per-mode=symbol] {\metre} & 10\%
  \\
  $L_\text{total}$ & $L_\text{total}$ > 0$ & 0 \leq $L_\text{total}$ \leq $L_{\text{max}}$ & 10 MWh & 10\%
  \\
  \bottomrule
\end{longtable*}
\end{table}

\noindent 
\begin{description}
\item[(*)] {the system able to convert MWh to KWh}
\end{description}

\begin{table}[!h]
\caption{Specification Parameter Values} \label{TblSpecParams}
\renewcommand{\arraystretch}{1.2}
\noindent \begin{longtable*}{l l} 
  \toprule
  \textbf{Var} & \textbf{Value} \\
  \midrule 
  $C_r$ & TBD\\
  $C_p$ & TBD\\
  \bottomrule
\end{longtable*}
\end{table}

\subsubsection{Properties of a Correct Solution} \label{sec_CorrectSolution}

\noindent
A correct solution must exhibit {fill in the details}.  {These
  properties are in addition to the stated requirements.  There is no need to
  repeat the requirements here.  These additional properties may not exist for
  every problem.  Examples include conservation laws (like conservation of
  energy or mass) and known constraints on outputs, which are usually summarized
  in tabular form.  A sample table is shown in Table~\ref{TblOutputVar}}

\begin{table}[!h]
\caption{Output Variables} \label{TblOutputVar}
\renewcommand{\arraystretch}{1.2}
\noindent \begin{longtable*}{l l} 
  \toprule
  \textbf{Var} & \textbf{Physical Constraints} \\
  \midrule 
  $A_i$ & Under data center computational power capable size
  \\
  \bottomrule
\end{longtable*}
\end{table}


\section{Requirements}

{The requirements refine the goal statement.  They will make heavy use of
  references to the instance models.}

This section provides the functional requirements, the business tasks that the
software is expected to complete, and the nonfunctional requirements, the
qualities that the software is expected to exhibit.

\subsection{Functional Requirements}

\noindent \begin{itemize}

\item[R\refstepcounter{reqnum}\thereqnum \label{R_Inputs}:] {Requirements for the inputs that are supplied by the user.  This information has to be explicit.}

\item[R\refstepcounter{reqnum}\thereqnum \label{R_OutputInputs}:] {The program shall notify user if an input is out of bounds.}

\item[R\refstepcounter{reqnum}\thereqnum \label{R_Calculate}:] {User should keep input validate data type.}

\item[R\refstepcounter{reqnum}\thereqnum \label{R_VerifyOutput}:]
  {The program should able to converting units while minimizing cost.}

\item[R\refstepcounter{reqnum}\thereqnum \label{R_Output}:] {The program should keep all decision variables under their constrains.}
\item[R\refstepcounter{reqnum}\thereqnum \label{R_VerifyOutput}:]
  {The program output the optimal plan for end users.}
  \item[R\refstepcounter{reqnum}\thereqnum \label{R_VerifyOutput}:]
  {The output should allocate the computing power of data centers with their capable size.}
\end{itemize}


\subsection{Nonfunctional Requirements}

{List your nonfunctional requirements.  You may consider using a fit
  criterion to make them verifiable.}
{The goal is for the nonfunctional requirements to be unambiguous, abstract
  and verifiable.  This isn't easy to show succinctly, so a good strategy may be
to give a ``high level'' view of the requirement, but allow for the details to
be covered in the Verification and Validation document.}
{An absolute requirement on a quality of the system is rarely needed.  For
  instance, an accuracy of 0.0101 \% is likely fine, even if the requirement is
  for 0.01 \% accuracy.  Therefore, the emphasis will often be more on
  describing now well the quality is achieved, through experimentation, and
  possibly theory, rather than meeting some bar that was defined a priori.}
{You do not need an entry for correctness in your NFRs.  The purpose of the
  SRS is to record the requirements that need to be satisfied for correctness.
  Any statement of correctness would just be redundant. Rather than discuss
  correctness, you can characterize how far away from the correct (true)
  solution you are allowed to be.  This is discussed under accuracy.}

\noindent \begin{itemize}

\item[NFR\refstepcounter{nfrnum}\thenfrnum \label{NFR_Accuracy}:]
  \textbf{Accuracy} {The accuracy of the computed
    solutions should meet the level needed for engineering application.}

\item[NFR\refstepcounter{nfrnum}\thenfrnum \label{NFR_Usability}:] \textbf{Usability}
  {The program shall not have a user interface but will clearly shows the output and data will easy to copy and read.}

\item[NFR\refstepcounter{nfrnum}\thenfrnum \label{NFR_Maintainability}:]
  \textbf{Maintainability} {The time complexity of this program should be O(n).}

\item[NFR\refstepcounter{nfrnum}\thenfrnum \label{NFR_Portability}:]
  \textbf{Portability} {The program should be easily integrated with another software program.}


\end{itemize}

\section{Likely Changes}    

\noindent \begin{itemize}

\item[LC\refstepcounter{lcnum}\thelcnum\label{LC_meaningfulLabel}:] {The likely changes for the program should be adding more renewable supply options.}
\item[LC\refstepcounter{lcnum}\thelcnum\label{LC_meaningfulLabel}:] {The program operating the analysis under minimum cost, it may have potential process analysis under minimum CO2 emission.}
\item[LC\refstepcounter{lcnum}\thelcnum\label{LC_meaningfulLabel}:] {The likely changes for the program should be more flexible for more operating systems.}
\end{itemize}

\section{Unlikely Changes}    

\noindent \begin{itemize}

\item[LC\refstepcounter{lcnum}\thelcnum\label{LC_meaningfulLabel}]:{The basic design concept of allocate computing instead of power station will unlikely to change.}

\end{itemize}
\section{Traceability Matrices and Graphs}

{The purpose of the traceability matrices is to provide easy references on what
has to be additionally modified if a certain component is changed.  Every time a
component is changed, the items in the column of that component that are marked with an ``X'' may have to be modified as well.  Table~\ref{Table:trace} shows the
dependencies of theoretical models, general definitions, data definitions, and instance models with each other.  Table~\ref{Table:R_trace} shows the dependencies of theoretical models,
general definitions, data definitions, instance models, and likely changes on the assumptions.}


\begin{table}[h!]
    \centering
    \begin{tabular}{|l|l|l|l|l|l|l|}
    \hline
        ~ & T1 & T2 & T3 & GD1 & DD1 & IM1 \\ \hline
        T1 & ~ & x & x & ~ & ~ & ~ \\ \hline
        T2 & ~ & ~ & ~ & ~ & x & ~ \\ \hline
        T3 & ~ & ~ & ~ & ~ & x & ~ \\ \hline
        GD1 & ~ & x & ~ & ~ & ~ & ~ \\ \hline
        DD1 & ~ & ~ & x & ~ & ~ & ~ \\ \hline
        IM1 & ~ & ~ & x & ~ & ~ & ~ \\ \hline
    \end{tabular}

\caption{Traceability Matrix Showing the Connections Between Items of Different Sections}
\label{Table:trace}
\end{table}

\begin{table}[!ht]
    \centering
    \begin{tabular}{|l|l|l|l|l|l|}
    \hline
        ~ & A1 & A2 & A3 & A4 & A5 \\ \hline
        T1 & ~ & ~ & ~ & ~ & ~ \\ \hline
        T2 & x & ~ & x & ~ & ~ \\ \hline
        T3 & ~ & ~ & ~ & x & x \\ \hline
        GD1 & ~ & x & ~ & ~ & ~ \\ \hline
        DD1 & ~ & x & ~ & ~ & ~ \\ \hline
        IM1 & ~ & ~ & ~ & ~ & ~ \\ \hline
    \end{tabular}
\caption{Traceability Matrix Showing the Connections Between Assumptions}
\label{Table:R_trace}
\end{table}

\section{Values of Auxiliary Constants}

This section contains nothing at this moment for minimization analysis.


\bibliographystyle {unsrt}
\bibliography{references}

\end{document}